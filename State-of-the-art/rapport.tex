\documentclass{rapport}
\usepackage[utf8]{inputenc}
\usepackage[T1]{fontenc} 
\usepackage{pifont} % Pour les symboles appelés par la macro \ding
\usepackage{url} % Comme son nom l'indique, pour les url...
\usepackage{amssymb}
\usetikzlibrary{positioning} % Bibliothèque tikz pour positionner des nœuds relativement à d'autres

\usepackage[colorlinks, citecolor=red!60!green, linkcolor=blue!60!green, urlcolor=magenta]{hyperref} % Pour que les liens soient cliquables. Les options permettent de mettre les liens en couleur.

\usepackage{mathrsfs}
\usepackage{pgfplots}
    
\usepackage{algorithm}
\usepackage{algo}
\usepackage{colorationSyntaxique}


% Pour un rapport en français 
\usepackage[french]{babel} % Commenter pour un rapport en anglais
\renewcommand\bibsection{\section*{Bibliographie}} % Commenter pour un rapport en anglais

% \englishTitlePage % Décommenter pour une page de titre en anglais


\pagestyle{fancy}
\renewcommand{\sectionmark}[1]{\markboth{\thesection.\ #1}{}}
\fancyfoot{}

\fancyhead[LE]{\textsl{\leftmark}}
\fancyhead[RE, LO]{\textbf{\thepage}}
\fancyhead[RO]{\textsl{\rightmark}}

\def\Latex{\LaTeX\xspace}
\def\etc{\textit{etc.}\xspace}



\title{Extraction automatique de ContextMaps dans des architectures micro-services}
\author{Thomas FARINEAU, Léo KITABDJIAN, Mohamed MAHJOUB}
\supervisor{Philippe COLLET}
\date{Second semestre de l'année 2023-2024}

 \universityname{Université Côte d'Azur} % Nom de l'université.
 \type{PER} % Type de document
 \formation{SI5 AL / M2} % Nom de la formation

% Retrouver les autres options possibles dans le document rapport.pdf

\begin{document}

  \maketitle

    \begin{abstract}
        \begin{itemize}
            \item ANALYSE DE L'EXISTANT \begin{itemize}
                \item DDD : Domain-Driven Design \begin{itemize}
                    \item Analyse de l'intégration de principes DDD dans Context Mapper et dans Context Mapper Discovery.
                    \item Identification des mécanismes de modélisation DDD pris en charge.
                    \item Évaluation de la conformité aux bonnes pratiques DDD.
                \end{itemize}
                \item Context Mapper ( https://contextmapper.org/docs/home/ ) \begin{itemize}
                    \item Présentation générale de Context Mapper
                    \item Explication des principaux concepts et fonctionnalités de Context Mapper.
                    \item Évaluation de la documentation existante pour déterminer la clarté et l'exhaustivité des informations fournies.
                \end{itemize}
                \item Context Mapper Discovery ( https://contextmapper.org/docs/reverse-engineering/ ) \begin{itemize}
                    \item Présentation de Context Mapper Discovery
                    \item Identification des types de modèles que Context Mapper Discovery peut générer à partir de code source existant.
                \end{itemize}
                \item Rétro-ingénierie \begin{itemize}
                    \item Compréhension approfondie du processus de rétro-ingénierie appliqué par Context Mapper Discovery.
                    \item Évaluation de la précision des résultats obtenus par la rétro-ingénierie.
                \end{itemize}
            \end{itemize} 
            \item BENCHMARKING

            \item Révision de la littérature scientifique (mettre des sources) \begin{itemize}
                \item https://stefan.kapferer.ch/2021/12/17/java-spektrum-article/ ( Explication du DSL + Liaison avec context mapper discovery ) Papier en Allemand
                \item Aller voir de la lecture sur du DDD (choisir des papiers)
                \item Aller voir de la lecture sur de la Rétro-Ingénierie 
                \item Aller voir des papiers de recherche sur des "concurrents à context Mapper Discovery (ou d'autres alternatives) 
            \end{itemize}
        \end{itemize}
    \end{abstract}

  \clearpage
  \tableofcontents

  \clearpage

\end{document}
